\section{Random Facts}

This section is a collection of facts we discovered about xtp and ctp which 
should be included in the manual at some point, but lack proper background.



\subsection{xqmm}

The cutoffs used should not exceed the dimension of the cell, at least for a non 
orthogonal unit cell. XQMM throws an error if your cutoff is larger than the 
box, but it does not take the extension of the molecule into accout, so often 
you may still have overlap.

XQMMM takes the segment coordinates from the MPS\textunderscore Files so be vary 
careful which MPS\textunderscore File you put in.


\subsection{EWALD}


\begin{itemize}
\item Use pewald3d, as a calculator. I am not sure what the rest is for. All 
still use the ewald options. The shape factor massively influences the results. 
For bulk systems "none" is the option of choice. Other options are "xyslab", 
"sphere", "cube" but I do not know what they do.
\item The induction cutoff should hardly ever exceed 3nm, because the 
calculation is expensive
\item If you want to use induction, you befor have to run ewdbgpol calculator 
and specify polar\textunderscore top.bgp in the options file for ewald. All the 
other parameters should be the same in ewdbpol and pewald3d
\end{itemize}

\subsection{GW-BSE}

There is a wide range of different approximations for $GW$-BSE. Here I try to 
summarize common abbreviations and methods and explain what our code does. This 
is not complete and certainly has some mistakes in it. 



\begin{itemize}
 \item COHSEX: RPA is only calculated for $\omega=0$. This amounts to 
$\varepsilon(vect{r},vect{r'},\omega)=\varepsilon(vect{r},vect{r'})$. This 
is also called the static approximation to $GW$
 \item Plasmon Pole model; RPA is calculated moslty twice, once at $\omega=0$ 
and $\omega=\omega_0$ , then these values are used to fit an analytic model to 
interpolate $\varepsilon(vect{r},vect{r'},\omega)$.
 \item Imaginary frequency integration, $\varepsilon$ is numerically integrated 
along the Imaginary frequency axis. This is done because $\varepsilon$ is much 
smoother along the Imaginary axis and thus requires less functional evaluations. 
Afterwards $\varepsilon$ is extended to real frequencies via analytic 
continuation.  
\end{itemize}

The calculation of $GW$. In VOTCA we also have the \textbf{shift} option. This 
is commonly called a scissor operator. This allows you to shift the unoccuppied 
KS-orbitals up in energy, making the resulting spectrum closer to the $GW$ 
spectrum. This is often a better starting point for the self-consistent 
evaluations.

\begin{itemize}
 \item $G_0W_0$. $G$ and $\varepsilon$ are calculated once from DFT results. $W$ 
is evaluated once from that. Then energies $ \varepsilon_i^{GW}$ are corrected 
via:
 \begin{equation}
   \varepsilon_i^{GW}=\varepsilon_i^{KS}+Z_i\bra{\phi^{KS}_i} 
\Sigma(\varepsilon_i^{KS})-V_{xc} \ket{\phi^{KS}_i}
 \end{equation}
 
 This is not implemented in VOTCA, because $GW_0$ is nearly as fast. 
 
 \item $GW_0$. $\varepsilon$ is calculated once from DFT results. $W$ is 
evaluated once from that. Then $\Sigma$ is calculated. The resulting energies 
are fed back into $\Sigma$, until self-consistency is achieved. This is denoted 
\textbf{fixed} in VOTCA.
 \begin{equation}
   \varepsilon_i^{GW}=\varepsilon_i^{KS}+\bra{\phi^{KS}_i} 
\Sigma(\varepsilon_i^{GW})-V_{xc} \ket{\phi^{KS}_i}
 \end{equation} 
 
 
 \item $scQPGW$. As $GW_0$, but after $\varepsilon_i^{GW}$ are converged, these 
energies are used to recalculate  $\varepsilon$ and then $W$, this is repeated 
until self-consistency is achieved. This is denoted \textbf{iterate} in VOTCA. 
This converges the ``eigenvalues'' of QP particles but along the 
$\ket{\phi^{GW}}$ states
 
 \item $scGW$. As $scQPGW$ but instead of correcting only the energies, the full 
$Sigma$ matrix is calculated and the eigenvalue problem for the QP is set up and 
solved and the whole system is then self-consistently solved in 
$\ket{\phi^{GW}}$ states and not in   $\ket{\phi^{KS}}$. This fully converges 
the eigenvalues and eigenvectors of the QP particles in the space spanned by 
$\ket{\phi^{KS}}$, e.g. $\ket{\phi^{GW}}$ are linear combinations of 
$\ket{\phi^{KS}}$. This is reported to be unstable because the vertex 
corrections are important now. This is at the moment implemented in VOTCA.
 

\end{itemize}

The calculations in BSE

\begin{itemize}
 \item 
\end{itemize}


