\section{Conjugated segments}
\label{sec:xmlsegments}
The file describing hopping sites, or conjugated segments, is used by 
practically all programs and calculators. It links the coarse-grained trajectory 
(positions and orientations of rigid fragments) and quantum-mechanical 
descriptions of all conjugated segments. The description of this \xml file 
(\xmlmap) is given in table \ref{tab:segments}. An example for is 
shown in listing~\ref{list:segments}.
%
\begin{table}[h]
% \caption{Description of conjugated segments (\xmlmap).} 
\label{tab:segments}
\rowcolors{1}{invisiblegray}{white} {\footnotesize 
\input{reference/xml/segments.xml} }
\end{table}
%
\lstdefinelanguage{MXML} {
   basicstyle=\ttfamily\scriptsize,
   sensitive=true,
   morecomment=[s][\color{gray}\rmfamily\itshape]{<!--}{-->}, 
   showstringspaces=false,
   numberstyle=\scriptsize,
   numberblanklines=true,
   showspaces=false,
   breaklines=true,
   showtabs=false,
   alsoletter={:},
   keywords = [1]
{segments,segment,coordinates,orbitals,basisset,torbital,reorg_charging,
reorg_discharging,qneutral,qcharged,energy,beadconj,molname,name,map,weights},  
 
  keywordstyle={[1]\color{blue}},
}
%
\lstinputlisting[
 language=MXML,
 label=list:segments,
 caption={\small \xml file describing \slink{segments}{conjugated segments}. 
Note that the mapping and weights for each segment are separated by a colon. 
}]%
{./input/dcv2t/segments.xml}


