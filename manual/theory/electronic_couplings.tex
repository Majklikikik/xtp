\section{Transfer integrals }
\label{sec:transfer_integrals}

The electronic transfer integral\index{electronic coupling}\index{transfer integral|see{electronic coupling}} element $J_{ij}$ entering the Marcus rates in \equ{marcus} is defined as
\begin{equation}
   J_{ij} = \left\langle \phi_i \left\vert \hat{H} \right\vert \phi_j \right\rangle ,
\label{equ:TI}
\end{equation}
where $\phi_i$ and $\phi_j$ are diabatic wavefunctions, localized on molecule $i$ and $j$ respectively, participating in the charge transfer, and $\hat{H}$ is the Hamiltonian of the formed dimer. Within the frozen-core approximation, the usual choice for the diabatic wavefunctions $\phi_i$\index{diabatic states} is the highest occupied molecular orbital (HOMO) in case of hole transport, and the lowest unoccupied molecular orbital (LUMO) in the case of electron transfer, while $\hat{H}$ is an effective single particle Hamiltonian, e.g. Fock or Kohn-Sham operator of the dimer. As such, $J_{ij}$ is a measure of the strength of the electronic coupling of the frontier orbitals of monomers mediated by the dimer interactions. 

Intrinsically, the transfer integral is very sensitive to the molecular arrangement, i.e. the distance and the mutual orientation of the molecules participating in charge transport. Since this arrangement can also be significantly influenced by static and/or dynamic disorder~\cite{baessler_charge_1993,troisi_charge-transport_2006,troisi_charge_2009,mcmahon_organic_2010,vehoff_charge_2010},
it is essential to calculate $J_{ij}$ explicitly for each hopping pair within a realistic morphology. Considering that the number of dimers for which \equ{TI} has to be evaluated is proportional to the number of molecules times their coordination number, computationally efficient and at the same time quantitatively reliable schemes are required.

\subsection{Projection of monomer orbitals on dimer orbitals (DIPRO)}
\label{sec:dipro}
An approach for the determination of the transfer integral that can be used for 
any single-particle electronic structure method (Hartree-Fock, DFT, or 
semiempirical methods) is based on the projection of monomer orbitals on a 
manifold of explicitly calculated dimer orbitals. This dimer projection (DIPRO) 
technique including an assessment of computational parameters such as the basis 
set, exchange-correlation functionals, and convergence criteria is presented in 
detail in ref.~\cite{baumeier_density-functional_2010}. A brief summary of the 
concept is given below.

We start from an effective Hamiltonian~\footnote{we use following notations: $a$ 
- number, $\veca$ - vector, $\matr{A}$ - matrix, $\oper{A}$ - operator}
%
\begin{equation}
  \oper{H}^\text{eff} = \sum_i \epsilon_i \oper{a}_i^\dagger \oper{a}_i + 
\sum_{j \neq i} J_{ij} \oper{a}_i^\dagger \oper{a}_j + c.c.
  \label{equ:dipro_eq1}
\end{equation}
%
where $\oper{a}_i^\dagger$ and $\oper{a}_i$ are the creation and annihilation 
operators for a charge carrier located at the molecular site $i$.
The electron site energy is given by $\epsilon_i$, while $J_{ij}$  is the 
transfer integral between two sites $i$ and $j$. We label their frontier 
orbitals (HOMO for hole transfer, LUMO for electron transfer) $\phi_i$ and 
$\phi_j$, respectively. Assuming that the frontier orbitals of a dimer 
(adiabatic energy surfaces) result exclusively from the interaction of the 
frontier orbitals of monomers, and consequently expand them in terms of $\phi_i$ 
and $\phi_j$. The expansion coefficients, $\vect{C}$, can be determined by 
solving the secular equation
%
\begin{equation}
  (\matr{H} - E \matr{S})\vect{C} = 0
  \label{equ:dipro_eq2}
\end{equation}
%
where $\matr{H}$ and $\matr{S}$ are the Hamiltonian and overlap matrices of the 
system, respectively. 
%
%Since it is easier to work in matrix form, the following
%equation also holds (equation (\ref{eq:dipro_eq2}) in matrix form):
%
%\begin{equation}
% \matr{H}\matr{U} = \matr{S}\matr{U}\matr{E}
%  \label{eq:dipro_eq3}
% \end{equation}
%
These matrices can be written explicitly as
%
\begin{equation}
% \begin{aligned}
  \matr{H} = 
  \begin{pmatrix}
    e_i    &  H_{ij} \\
    H_{ij}^* &  e_j  
  \end{pmatrix} \hspace{2cm}
  \matr{S} = 
  \begin{pmatrix}
    1    &  S_{ij} \\
    S_{ij}^* &  1  
  \end{pmatrix}
%  \end{aligned}
  \label{equ:dipro_eq3}
\end{equation}
%
with 
%
\begin{equation}
 \begin{aligned}
  e_i &= \bra{\phi_i}\oper{H} \ket{\phi_i} \hspace{2cm}  H_{ij} = 
\bra{\phi_i}\oper{H} \ket{\phi_j}\\
  e_j &= \bra{\phi_j}\oper{H} \ket{\phi_j} \hspace{2cm}  S_{ij} = \bra{\phi_j} 
\phi_j\rangle %S 
 \end{aligned}
  \label{equ:dipro_eq4}
\end{equation}
The matrix elements $e_{i(j)}$, $H_{ij}$, and $S_{ij}$ entering \equ{dipro_eq3} 
can be calculated via projections on the dimer orbitals (eigenfunctions of 
$\hat{H}$) $\left\{\ket{\phi^\text{D}_n}\right\}$ by inserting $\oper{1} = 
\sum_n \ket{\phi^\text{D}_n}\bra{\phi^\text{D}_n}$ twice. We exemplify this 
explicitly for $H_{ij}$ in the following
%
\begin{equation}
  H_{ij} = \sum_{nm}{\braket{\phi_i}{\phi^\text{D}_n} 
\bra{\phi^{D}_n}\hat{H}\ket{\phi^\text{D}_m}\braket{\phi^\text{D}_m}{\phi_j}} .
  \label{eq:dipro_eq16}
\end{equation}
%
The Hamiltonian is diagonal in its eigenfunctions, 
$\bra{\phi^\text{D}_n}\oper{H}\ket{\phi^\text{D}_m} = E_n \delta_{nm}$. 
Collecting the projections of the frontier orbitals  $\ket{\phi_{i(j)}}$ on the 
$n$-th dimer state $\left(\vect{V}_{(i)}\right)_n= 
\braket{\phi_i}{\phi^\text{D}_n}$ and 
$\left(\vect{V}_{(j)}\right)_n=\braket{\phi_j}{\phi^\text{D}_n}$ respectively, 
into vectors we obtain

\begin{equation}
   H_{ij} = \vect{V}_{(i)} \matr{E}   \vect{V}_{(j)}^\dagger .
  \label{eq:dipro_eq17}
\end{equation}
%
What is left to do is determine these projections $\vect{V}_{(k)}$. In all 
practical calculations the molecular orbitals are expanded in basis sets of 
either plane waves or of localized atomic orbitals $\ket{\varphi_\alpha}$. We 
will first consider the case that the calculations for
the monomers are performed using a counterpoise basis set that is commonly used 
to deal with the basis set superposition error (BSSE). The basis set of 
atom-centered orbitals of a monomer is extended to the one of the dimer by 
adding the respective atomic orbitals at virtual coordinates of the second 
monomer. We can then write the respective expansions as

\begin{equation}
 %\begin{aligned}
  \ket{\phi_{k}} = \sum_{\alpha} \lambda^{(k)}_\alpha \ket{\varphi_\alpha} 
\hspace{1cm}\text{and}\hspace{1cm}
  \ket{\phi^\text{D}_n} = \sum_{\alpha} D^{(n)}_\alpha \ket{\varphi_\alpha}
  \label{eq:dipro_eq18}
\end{equation}
%
where $k=i,j$. The projections can then be determined within this common basis 
set as

 \begin{equation}
  \begin{aligned}
     \left(\vect{V}_k\right)_n=\braket{\phi_k}{\phi^\text{D}_n} = \sum_{\alpha} 
\lambda^{(k)}_{\alpha} \bra{\alpha} \sum_{\beta} D^{(n)}_{\beta} \ket{\beta} = 
     \vect{\boldsymbol{\lambda}}_{(k)}^\dagger \matr{\mathcal{S}} 
\vect{D}_{(n)} 
%     %\\
% %    \braket{B|i} = \sum_{\alpha} B_{\alpha} \bra{\alpha}
% %    \sum_{\beta} D^{(i)}_{\beta} \ket{\beta} = 
% %    \vect{B}^\dagger \matr{S} \vect{D}^{(i)} \\
  \end{aligned}
   \label{eq:dipro_eq19}
 \end{equation}
where $\matr{\mathcal{S}}$ is the overlap matrix of the atomic basis functions. 
This allows us to finally write the elements of the Hamiltonian and overlap 
matrices in \equ{dipro_eq3} as:

 \begin{equation}
  \begin{aligned}
     H_{ij} &= \vect{\boldsymbol{\lambda}}_{(i)}^\dagger \matr{\mathcal{S}} 
\matr{D} \matr{E} \matr{D}^\dagger \matr{\mathcal{S}}^\dagger 
\vect{\boldsymbol{\lambda}}_{(j)}  \\
     S_{ij} &= \vect{\boldsymbol{\lambda}}_{(i)}^\dagger \matr{\mathcal{S}} 
\matr{D}  \matr{D}^\dagger \matr{\mathcal{S}}^\dagger 
\vect{\boldsymbol{\lambda}}_{(j)} 
  \end{aligned}
   \label{eq:dipro_eq20}
 \end{equation}
%
Since the two monomer frontier orbitals that form the basis of this expansion 
are not orthogonal in general ($\matr{S} \neq \matr{1}$), it is necessary to 
transform \equ{dipro_eq2} into a standard eigenvalue problem of the form
%
\begin{equation}
  \matr{H}^{\mathrm{eff}} \vect{C}^{\mathrm{eff}} =   E \vect{C}^{\mathrm{eff}} 
  \label{eq:dipro_eq7}
\end{equation}
%
to make it correspond to \equ{dipro_eq1}. According to L\"owdin such a 
transformation can be achieved by
%
\begin{equation}
  \matr{H^\mathrm{eff}} = \matr{S}^{\left. {-1} \middle/ {2} \right.}
  \matr{H}\matr{S}^{\left. {-1} \middle/ {2} \right.}.
  \label{eq:dipro_eq9}
\end{equation}
%
This then yields an effective Hamiltonian matrix in an orthogonal basis, and its 
entries can directly be identified with the site energies $\epsilon_i$ and 
transfer integrals $J_{ij}$:
%
\begin{equation}
 \begin{aligned}
  \matr{H}^{\mathrm{eff}} &= 
    \begin{pmatrix}
      e_i^{\mathrm{eff}}    &  H_{ij}^\mathrm{eff} \\
      H_{ij}^{*,\mathrm{eff}}   &  e_j^\mathrm{eff}  
    \end{pmatrix} =
    \begin{pmatrix}
      \epsilon_i    &  J_{ij} \\
      J_{ij}^*      &  \epsilon_j  
    \end{pmatrix} 
 \end{aligned}
  \label{eq:dipro_eq11}
\end{equation}

 \begin{figure}[htb]
     \center
     \includegraphics[width=\linewidth]{fig/idft_flow/schemes_all}
     \caption{Schematics of the DIPRO method. (a) General workflow of the 
projection technique. (b) Strategy of the efficient noCP+noSCF implementation, 
in which the monomer calculations are performed independently form the dimer 
configurations (noCP), using the \calc{edft} \calculator. The dimer Hamiltonian 
is subsequently constructed based on an initial guess formed from monomer 
orbitals and only diagonalized once (noSCF) before the transfer integral is 
calculated by projection. This second step is performed by the \calc{idft} 
\calculator. }
     \label{fig:dipro_scheme}
 \end{figure}



\subsection{DFT-based transfer integrals using DIPRO}
\label{sec:dft}
\index{electronic coupling!DFT}
% While the use of the semiempirical ZINDO method provides an efficient 
on-the-fly technique to determine electronic coupling elements, it is not 
generally applicable to all systems. For instance, its predictive capacity with 
regards to atomic composition and localization behavior of orbitals within more 
complex structures is reduced. Moreover, transition- or semi-metals are often 
not even parametrized. In this case, {\it ab-initio} based approaches, e.g., 
density-functional theory can remedy the 
situation~\cite{huang_intermolecular_2004,huang_validation_2005,
valeev_effect_2006,yin_balanced_2006,yang_theoretical_2007,
baumeier_density-functional_2010}. 

The calculation of one electronic coupling element based on DFT using the \dipro 
method requires the overlap matrix of atomic orbitals $\matr{\mathcal{S}}$, the 
expansion coefficients for monomer $\vect{\boldsymbol{\lambda}}_{(k)} = \{ 
\lambda_\alpha^{(k)}\}$ and dimer orbitals $\vect{D}_{(n)} = \{ D^{(n)}_{\alpha} 
\}$, as well as the orbital energies $E_{n}$ of the dimer are required as input. 
In practical situations, performing self-consistent quantum-chemical 
calculations for each individual monomer and one for the dimer to obtain this 
input data is extremely demanding. Several simplifications can be made to reduce 
the computational effort, such as using non-Counterpoise basis sets for the 
monomers (thereby decoupling the monomer calculations from the dimer run) and 
performing only a single SCF step in a dimer calculation starting from an 
initial guess formed from a superposition of monomer orbitals. This 
''noCP+noSCF'' variant of \dipro is shown in \fig{dipro_scheme}(a) and 
recommended for production runs. 
A detailed comparative study of the different variants can be found 
in~\cite{baumeier_density-functional_2010}.

The code currently contains supports evaluation of transfer integrals from 
quantum-chemical calculations performed with the \gaussian, \orca, and 
\nwchem packages. The interfacing procedure consists of three main steps: 
generation of input files for monomers and dimers, performing the actual 
quantum-chemical calculations, and calculating the transfer integrals. 

\subsubsection{Monomer calculations}
\label{sec:edft}
First, \slink{sec:segments}{hopping sites} and a 
\slink{sec:neighborlist}{neighbor list} need to be generated from the atomistic 
topology and trajectory and written to the \sqlstate file. Then the parallel 
\calc{edft} \calculator manages the calculation of the monomer properties 
required for the determination of electronic coupling elements. Specifically, 
the individual steps it performs are:
%
\begin{enumerate}
\item Creation of a job file containing the list of molecules to be calculated 
with DFT 
\votcacommand{Writing job file for \calc{edft}}{\xtpparallel \opt \xmloptions 
\sql \sqlstate \exe \calc{edft} \job \wrt }
\item Running of all jobs in job file 
\votcacommand{Running all \calc{edft} jobs}{\xtpparallel \opt \xmloptions \sql 
\sqlstate \exe \calc{edft} \job \run }
which includes
\begin{itemize}
\item creating the input files for the DFT calculation (using the package 
specified in \xmloptions) in the directory 
\begin{verbatim}
OR_FILES/package/frame_F/mol_M
\end{verbatim}
where {\tt F} is the index of the frame in the trajectory, {\tt M} is the index 
of a molecule in this frame,
\item executing the DFT run, and
\item after completion of this run, parsing the output (number of electrons, 
basis set, molecular orbital expansion coefficients), and saving it in 
compressed form to 
\begin{verbatim}
 OR_FILES/molecules/frame_F/molecule_M.orb 
\end{verbatim}
\end{itemize}
\end{enumerate}


\subsubsection{Calculating the transfer integrals}
\label{sec:idft}
After the momomer calculations have been completed successfully, the respective 
runs for dimers from the neighborlist can be performed using the parallel 
\calc{idft} \calculator, which manages the DFT runs for the hopping pairs and 
determines the coupling element using \dipro. Again, several steps are required:
\begin{enumerate}
\item Creation of a job file containing the list of pairs to be calculated with 
DFT 
\votcacommand{Writing job file for \calc{idft}}{\xtpparallel \opt \xmloptions 
\sql \sqlstate \exe \calc{idft} \job \wrt }
\item Running of all jobs in job file 
\votcacommand{Running all \calc{idft} jobs}{\xtpparallel \opt \xmloptions \sql 
\sqlstate \exe \calc{idft} \job \run }
which includes
\begin{itemize}
\item creating the input files (including the merged guess for a noSCF 
calculation, if requested) for the DFT calculation (using the package specified 
in \xmloptions) in the directory 
\begin{verbatim}
OR_FILES/package/frame_F/pair_M_N
\end{verbatim}
where {\tt M} and {\tt N} are the indices of the molecules in this pair,
\item executing the DFT run, and
\item after completion of this run, parsing the output (number of electrons, 
basis set, molecular orbital expansion coefficients and energies, atomic orbital 
overlap matrix), and saving the pair information in compressed form to
\begin{verbatim}
 OR_FILES/pairs/frame_F/pair_M_N.orb 
\end{verbatim}
\item loading the monomer orbitals from the previously saved {\tt *.orb files}.
\item calculating the coupling elements and write them to the job file
\end{itemize}
\item Reading the coupling elements from the job file and saving them to the 
\sqlstate file
\votcacommand{Saving \calc{idft} results from job file to 
\sqlstate}{\xtpparallel \opt \xmloptions \sql \sqlstate \exe \calc{idft} \job 
\rd }
\end{enumerate}






\subsection{ZINDO-based transfer integrals using MOO }
\label{sec:izindo}

\newcommand{\xtp}{MOO\xspace}
\index{electronic coupling!ZINDO}

An approximate method based on Zerner's Intermediate Neglect of Differential Overlap (ZINDO) has been described in Ref.~\cite{kirkpatrick_approximate_2008}. This semiempirical method is substantially faster than first-principles approaches, since it avoids the self-consistent calculations on each individual monomer and dimer. This allows to construct the matrix elements of the ZINDO Hamiltonian of the dimer from the weighted overlap of molecular orbitals of the two monomers. Together with the introduction of rigid segments, only a single self-consistent calculation on one isolated conjugated segment is required. All relevant molecular overlaps can then be constructed from the obtained molecular orbitals.

The main advantage of the molecular orbital overlap (\xtp) library is {\em fast} evaluation of electronic coupling elements. Note that \xtp is based on the ZINDO Hamiltonian which has limited applicability. The general advice is to first compare the accuracy of the \xtp method to the DFT-based calculations. 

\xtp can be used both in a \slink{prog:xtp_overlap}{standalone mode} and as an \calc{izindo} \calculator of \votcaxtp. 

Since \xtp constructs the Fock operator of a dimer from the  molecular orbitals of monomers by translating and rotating the orbitals of \slink{sec:segments}{rigid fragments}, the optimized geometry of all \slink{sec:segments}{conjugated segments} and the coefficients of the molecular orbitals are required as its input in addition to the state file (\sqlstate) with the \slink{sec:neighborlist}{neighbor list}. Coordinates are stored in \xyz files with four columns, first being the atom type and the next three atom coordinates. This is a standard \texttt{xyz} format without a header. Note that the atom order in the \xyz files can be different from that of the mapping files. The correspondence between the two is established in the \xmlcsg file. 

\attention{Izindo requires the specification of orbitals for hole and electron transport in \xmlcsg. They are the HOMO and LUMO respectively and can be retrieved from the \texttt{log} file from which the \orb file is generated. The number of \texttt{alpha electrons} is the HOMO, the LUMO is HOMO+1 } 

The calculated transfer integrals are immediately saved to the \sqlstate file.
\votcacommand{Transfer integrals from \calc{izindo}}{\xtprun \opt \xmloptions \sql \sqlstate \exe \calc{izindo} }



