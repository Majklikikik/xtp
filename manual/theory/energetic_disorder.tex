\section{Site energies}
\label{sec:site_energies}
A charge transfer reaction between molecules $i$ and $j$ is driven by the site 
energy\index{site energy} difference, $\Delta E_{ij} = E_i - E_j$. Since the  
transfer rate, $\omega_{ij}$, depends exponentially on $\Delta E_{ij}$ 
(see~\equ{marcus}) it is important to compute its distribution as accurately as 
possible.  The total site energy difference has contributions due to 
\slink{sec:ext_field}{externally applied electric field}, 
\slink{sec:ecoulomb}{electrostatic interactions}, polarization effects, and 
\slink{sec:internal_energy}{internal energy} differences. In what follows we 
discuss how to estimate these contributions by making use of first-principles 
calculations and polarizable force-fields.

\subsection{Externally applied electric field}
\label{sec:ext_field}
The contribution to the total site energy\index{site energy!external field} 
difference due to an external electric field $\vec{F}$ is given by $\Delta 
E_{ij}^\text{ext} = q {\vec{F} \cdot \vec{r}_{ij}}$, where $q=\pm e$ is the 
charge and $\vec{r}_{ij} = \vec{r}_i  - \vec{r}_j $ is a vector connecting 
molecules $i$ and $j$. For typical distances between small molecules, which are 
of the order  of $1\,\unit{nm}$, and moderate fields of $F<10^8\,\unit{V/m}$ 
this term is always smaller than $0.1\, \unit{eV}$.

\subsection{Internal energy}
\label{sec:internal_energy}

The contribution to the site energy difference due to different internal 
energies\index{site energy!internal} (see \fig{parabolas}) can be written as
\begin{equation}
 \Delta E_{ij}^\text{int}=
\Delta U_i - \Delta U_j = \left( U_{i}^{cC}-U_{i}^{nN}\right) - \left( 
U_{j}^{cC}-U_{j}^{nN}\right) \, ,
\label{equ:conformational}
\end{equation}
where $U_{i}^{cC(nN)}$ is the total energy of molecule $i$ in the charged 
(neutral) state and geometry.  $\Delta U_{i}$ corresponds to the adiabatic 
ionization potential (or electron affinity) of molecule $i$, as shown 
in~\fig{parabolas}. For one-component systems and negligible conformational 
changes $ \Delta E_{ij}^\text{int}=0$, while it is significant for 
donor-acceptor systems. 

Internal energies determined using quantum-chemistry need to be specified in 
\xmlmap. The values are written to the \sqlstate using the calculator 
\calc{einternal} (see also \slink{sec:eintramolecular}{intramolecular 
reorganization energy}):
\votcacommand{Internal energies}{\cmdeint}
