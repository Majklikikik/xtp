\subsection{Induction energy - the Thole model}
\label{sec:thole_model}
\index{Thole model}
\index{site energy!polarization}


If we in addition to the permanent set of multipole moments $\{Q_t^a\}$ allow 
for induced moments $\{\Delta Q_t^a\}$ and penalize their generation with a 
bilinear form (giving rise to a strictly positive contribution to the energy),
\begin{align}
U_\textrm{int} &=\frac{1}{2} \sum_A \Delta Q_t^a \eta_{tt'}^{aa'} \Delta 
Q_{t'}^{a'},
\end{align}
it can be shown that the induction contribution to the site energy evaluates to 
an expression where all interactions between induced moments have cancelled out, 
and interactions between permanent and induced moments are scaled down by 
$1/2$~\cite{stone_theory_1997}:
\begin{align}
U_{pu} = \frac{1}{2} \sum_A \sum_{B > A} \left[ \Delta Q_t^a T_{tu}^{ab} Q_u^b + 
\Delta Q_t^b T_{tu}^{ab} Q_u^a \right].
\label{equ:u_pu}
\end{align}
This term can be viewed as the second-order (induction) correction to the 
molecular interaction energy. The sets of $\{Q_t^a\}$ are solved for 
self-consistently via
\begin{align}
\Delta Q_t^a = - \sum_{B \neq A} \alpha_{tt'}^{aa'} T_{t'u}^{a'b} (Q_u^b + 
\Delta Q_u^b),
\label{equ:self_consistent_dQ}
\end{align}
where the polarizability tensors $\alpha_{tt'}^{aa'}$ are given by the inverse 
of $\eta_{tt'}^{aa'}$.

With eqs.~\ref{equ:self_consistent_dQ} and~\ref{equ:u_pu} we have at hand 
expressions that allow us to compute the induction energy contribution to site 
energies in an iterative manner based on a set of molecular distributed 
multipoles $\{Q_t^a\}$ and polarizabilities $\{\alpha_{tt'}^{aa'}\}$. We have 
drafted in the previous section how to obtain the former from a wavefunction 
decomposition or fitting scheme (GDMA, CHELPG). The $\{\alpha_{tt'}^{aa'}\}$ can 
be derived formally (or rather: read off) from a perturbative expansion of the 
molecular interaction. In this work we make use of the Thole 
model~\cite{thole_molecular_1981, van_duijnen_molecular_1998} as a 
semi-empirical approach to obtain the sought-after point polarizabilities in the 
local dipole approximation, that is, $[\alpha_{tt'}^{aa'}] = \alpha_{tt'}^{aa'} 
\delta_{t \beta} \delta_{t'\beta} \delta_{aa'}$, where $\beta \epsilon 
\{x,y,z\}$ references the dipole-moment component.

The Thole model is based on a modified dipole-dipole interaction, which can be 
reformulated in terms of the interaction of smeared charge densities. This has 
been shown to be necessary due to the divergent head-to-tail dipole-dipole 
interaction that otherwise results at small interseparations on the 
\AA~scale~\cite{applequist_atom_1972, thole_molecular_1981, 
van_duijnen_molecular_1998}. Smearing out the charge distribution mimics the 
nature of the QM wavefunction, which effectively guards against this unphysical 
polarization catastrophe. Since the point dipoles however only react 
individually to the external field, any correlation effects as were still 
accounted for in the $\{\alpha_{tt'}^{aa'}\}$ are lost, except perhaps those 
correlations that are due to the mere classical field interaction.

The smearing of the nuclei-centered multipole moments is obtained via a 
fractional charge density $\rho_f(\vec{u})$ which should be normalized to unity 
and fall off rapidly as of a certain radius $\vec{u} = \vec{u}(\vec{R})$. The 
latter is related to the physical distance vector $\vec{R}$ connecting two 
interacting sites via a linear scaling factor that takes into account the 
magnitude of the isotropic site polarizabilities $\alpha^a$. This isotropic 
fractional charge density gives rise to a modified potential
\begin{align}
 \phi(u) = -\frac{1}{4\pi\varepsilon_0} \int \limits_{0}^{u} \! 4\pi u' \rho(u') 
d\!u' 
 \label{equ:mod_potential}
\end{align}
We can relate the multipole interaction tensor $T_{ij \dots}$ (this time in 
Cartesian coordinates) to the fractional charge density in two steps: First, we 
rewrite the tensor in terms of the scaled distance vector $\vec{u}$,
\begin{align}
 T_{ij \dots }(\vec{R}) = f(\alpha^a \alpha^b) \ t_{ij 
\dots}(\vec{u}(\vec{R},\alpha^a \alpha^b)),
\end{align}
where the specific form of $f(\alpha^a \alpha^b)$ results from the choice of 
$u(\vec{R},\alpha^a \alpha^b)$. Second, we demand that the smeared interaction 
tensor $t_{ij \dots}$ is given as usual by the appropriate derivative of the 
potential in eq.~\ref{equ:mod_potential},
\begin{align}
 t_{ij \dots}(\vec{u}) = - \partial_{u_i} \partial_{u_j} \dots \phi(\vec{u}).
\end{align}
It turns out that for a suitable choice of $\rho_f(\vec{u})$, the modified 
interaction tensors can be rewritten in such a way that powers $n$ of the 
distance $R = |\vec{R}|$ are damped with a damping function 
$\lambda_n(\vec{u}(\vec{R}))$~\cite{ren_polarizable_2003}.

There is a large number of fractional charge densities $\rho_f(\vec{u})$ that 
have been tested for the purpose of giving best results for the molecular 
polarizability as well as interaction energies. Note how a great advantage of 
the Thole model is the exceptional transferability of the atomic 
polarizabilities to compounds not used for the fitting 
procedure~\cite{van_duijnen_molecular_1998}. In fact, for most organic 
molecules, a fixed set of atomic polarizabilities ($\alpha_C = 1.334$, $\alpha_H 
= 0.496$, $\alpha_N = 1.073$, $\alpha_O = 0.873$, $\alpha_S = 2.926$ \AA$^3$) 
based on atomic elements yields satisfactory results.

VOTCA implements the Thole model with an exponentially-decaying fractional 
charge density
\begin{align}
 \rho(u) = \frac{3a}{4\pi} \exp(-au^3),
\end{align}
where $\vec{u}(\vec{R},\alpha^a \alpha^b) = \vec{R} / (\alpha^a \alpha^b)^{1/6}$ 
and the smearing exponent $a=0.39$ (which can however be changed from the 
program options), as used in the AMOEBA force field~\cite{ren_polarizable_2003}.

Even though the Thole model performs very well for many organic compounds with 
only the above small set of element-based polarizabilities, conjugated molecules 
may require a more intricate parametrization. The simplest approach is to resort 
to scaled polarizabilities to match the effective molecular polarizable volume 
$V \sim \alpha_{x} \alpha_{y} \alpha_{z}$ as predicted by QM calculations (here 
$\alpha_x, \alpha_y, \alpha_z$ are the eigenvalues of the molecular 
polarizability tensor). The \toolref{molpol} \tool assists with this task, it 
self-consistently calculates the Thole polarizability for an input mps-file and 
optimizes (if desired) the atomic polarizabilities in the above simple manner.

\votcacommand{Generate Thole-type polarizabilites for a segment}{\cmdmolpol}

The electrostatic and induction contribution to the site energy is evaluated by 
the \calc{emultipole} \calculator. Atomistic partial charges for charged and 
neutral molecules are taken from mps-files (extended GDMA format) specified in 
\xmlmap. Note that, in order to speed up calculations for both methods, a 
cut-off radius (for the molecular centers of mass) can be given in \xmloptions. 
Threaded execution is advised.

\votcacommand{Electrostatic and induction corrections}{\cmdemlt}

Furthermore available are \calc{zmultipole}, which extends \calc{emultipole} to 
allow for an electrostatic buffer layer (loosely related to the z-buffer in 
OpenGL, hence the name) and anisotropic point polarizabilities. For the 
interaction energy of charged clusters of any user-defined composition (Frenkel 
states, CT states, ...), \calc{xqmultipole} can be used.

\votcacommand{Interaction energy of charged molecular clusters embedded in a 
molecular environment}{\cmdxqmult}
